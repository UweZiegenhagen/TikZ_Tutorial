%!TEX TS-program = Arara
% arara: pdflatex: {shell: yes}
\documentclass[14pt,ngerman]{beamer}

\usepackage[T1]{fontenc}
\usepackage{booktabs}
\usepackage{babel}
\usepackage{graphicx}
\usepackage{csquotes}
\usepackage{xcolor}

\author{Uwe Ziegenhagen}
\title{\textit{TikZ}ische Erlebnisse}
\subtitle{Dante Frühjahrstagung 2025}


\begin{document}

\begin{frame}

\maketitle

\end{frame}

\begin{frame}
\frametitle{Über diesen Vortrag}

\begin{itemize}
\item Habe diverse Dinge mit TikZ umgesetzt
\item Erkenntnis: TikZ ist \textit{echt} mächtig 
\item Programm für heute: 
\begin{itemize}
	\item Grundlagen
	\item Beispiele
\end{itemize}

\end{itemize}
\end{frame}

\section{Geschichte und Grundlagen} 

\begin{frame}
\frametitle{Geschichte}

\begin{itemize}
\item TikZ = TikZ ist kein Zeichenprogramm
\item TikZ = \enquote{Frontend} für PGF
\item Entwickler Till Tantau, Christian Feuersänger
\end{itemize}
\end{frame}



\begin{frame}
\frametitle{Grundlagen I}

Linien von a nach b
Bezier von c nach d via e und f

\end{frame}


\begin{frame}
\frametitle{Grundlagen II}

nodes

\end{frame}

\begin{frame}
\frametitle{Grundlagen III}

Koordinaten

\end{frame}








\end{document}